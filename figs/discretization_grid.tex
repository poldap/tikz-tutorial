% Mathematical macros needed (related to paper content)
\newcommand{\supp}[1]{\operatorname{supp}\left( #1 \right)}
\def\sigmax{\sigma_{\max}}
\def\dsigma{\tilde{\sigma}}
\renewcommand{\complement}[1]{#1^{\mathsf{c}}}

% Define scaling variable
\def\sca{0.73}
% Set perspective for 3D rendering
\tdplotsetmaincoords{70}{110}

\begin{tikzpicture}[tdplot_main_coords,scale=\sca]
	% Define variables for drawing 
    	% Margin distance in x and min and max for sigma
		    \def\xdist{3.5} \def\sigminn{0} \def\sigmaxn{4.5}
	    % Margin distance in sigma and min and max for spatial coordinates
		    \def\sigdist{-2} \def\xmin{-.5} \def\xmax{-6} \def\ymin{0.25} \def\ymax{3.5}
		    
	% Draw support of spatial masking function 
		% Limits of spatial coordinates
		    \draw[blue] plot coordinates {(\xmin,\sigdist,\ymin) (\xmax,\sigdist,\ymin) (\xmax,\sigdist,\ymax) (\xmin,\sigdist,\ymax) (\xmin,\sigdist,\ymin)};
		    \node[color=blue] at (\xmax,\sigdist-1,\ymax+.2) {$\supp{\mu}$};
		% Guiding lines (observe \foreach)
		    \foreach \x in {\xmin,\xmax} {
			    \foreach \y in {\ymin,\ymax}{
				    \draw[thin,dashed,blue] (\x,\sigdist,\y) -- (\x,0,\y);
			    }
		    }
		
	% Grid for discretization of the spatial component
		% Configure variables
		    \def\nbins{5} \def\binsextra{2}
		% Set PGF variables as mathematical expressions of other variables
		    \pgfmathsetmacro{\nbinsextrapo}{-\binsextra +1} \pgfmathsetmacro{\nbinsextra}{\nbins +\binsextra}
		    \pgfmathsetmacro{\xbynbins}{(\xmax-\xmin)/\nbins} \pgfmathsetmacro{\ybynbins}{(\ymax-\ymin)/\nbins}
		% Draw spatial grid
		% Observe \foreach extrapolation \n in {0, 1, ..., 5}
		    \foreach \n in {-\binsextra, \nbinsextrapo, ...,\nbinsextra}{
			    % X divisions
			        \draw[gray!40,thin] (\xmin + \n*\xbynbins,0,\ymin-\binsextra*\ybynbins) -- (\xmin + \n*\xbynbins,0,\ymin+\nbinsextra*\ybynbins);
			    % Y divisions
			        \draw[gray!40,thin] (\xmin-\binsextra*\xbynbins,0,\ymin + \n*\ybynbins) -- (\xmin+\nbinsextra*\xbynbins,0,\ymin + \n*\ybynbins);
		    }
		% Draw box around spatial grid
		    \draw[thin] (\xmin -\binsextra*\xbynbins,0,\ymin-\binsextra*\ybynbins) -- (\xmin+\nbinsextra*\xbynbins,0,\ymin-\binsextra*\ybynbins) -- (\xmin+\nbinsextra*\xbynbins,0,\ymin+\nbinsextra*\ybynbins) node[anchor=south] {\textcolor{gray}{sensor's grid}} --	(\xmin-\binsextra*\xbynbins,0,\ymin+\nbinsextra*\ybynbins) -- cycle;
		% Draw mask support on the same level as the sensor grid
		\draw[blue,thin,dashed] plot coordinates {(\xmin,0,\ymin) (\xmax,0,\ymin) (\xmax,0,\ymax) (\xmin,0,\ymax) (\xmin,0,\ymin)};
		
	% Draw sigma discretization grid
		% Limits of sigma regions (note path does not end until ;)
		    \draw[red] (.1+\xdist,0,0) -- (-.1+\xdist,0,0) -- % Beginning tick
			(\xdist,\sigminn,0)  -- node[midway,below]{$\left[0,\sigmax\right]$} (\xdist,\sigmaxn,0) -- % Margins
			(.1+\xdist,\sigmaxn,0) -- (-.1+\xdist,\sigmaxn,0); % Ending tick
		% Guiding lines
		    \draw[thin,dashed,red] (\xdist,0,0) -- (0,0,0);
		    \draw[thin,dashed,red] (\xdist,4.5,0) -- (0,4.5,0);
		% Line for marking discretization of sigma
		    \draw[gray,thin] (.25*\xdist,\sigminn,0) -- (.25*\xdist,\sigmaxn,0);
		% Set PGF variable to length of sigma 
 		    \pgfmathsetmacro{\lofsig}{\sigmaxn-\sigminn}
 		% Mark discretization of sigma in line
 		    \foreach \prop in {0, .15, .25, .333, .5, .75, 1}{
			    \draw (.1+.25*\xdist,\prop*\lofsig,0) -- (-.1+.25*\xdist,\prop*\lofsig,0);
 			}
 		% Mark set of sigmas aleph and its complement
     		\draw[red,thin] (.5*\xdist,\sigminn,0) -- node[midway,below]{\scriptsize $\aleph$} (.5*\xdist,\sigminn+.75*\lofsig,0);
     		\draw[red,thin] (.5*\xdist,\sigminn+.75*\lofsig,0) -- node[midway,below]{\scriptsize $\complement{\aleph}$} (.5*\xdist,\sigmaxn,0);
 		% Guiding lines for aleph and its complement
     		\foreach \prop in {0, .75, 1}{
    			% Divisions in sigma axis
     			\draw[red,thin] (.1+.5*\xdist,\prop*\lofsig,0) -- (-.1+.5*\xdist,\prop*\lofsig,0);
     			% Divisions in 2D Space
     			\draw[red,thin,dashed] (.5*\xdist,\prop*\lofsig,0) -- (0,\prop*\lofsig,0);
     		}
	
	% Draw discrete axis (m,n,k)
	    % Spatial axis (m,n)
	        \draw[thick,->] (\xmin-\binsextra*\xbynbins,0,\ymin-\binsextra*\ybynbins) -- (\xmin-\binsextra*\xbynbins+.5*\xbynbins,0,\ymin-\binsextra*\ybynbins) node[anchor=west]{\tiny $\mathbf{n}$};
	        \draw[thick,->] (\xmin-\binsextra*\xbynbins,0,\ymin-\binsextra*\ybynbins) -- (\xmin-\binsextra*\xbynbins,0,\ymin+.5*\ybynbins-\binsextra*\ybynbins) node[anchor=east]{\tiny $\mathbf{m}$};
	    % Third dimension k
	        \draw[thick,->] (.25*\xdist,0,0) -- (.25*\xdist,.08*\lofsig,0) node[anchor=north east]{\tiny $\mathbf{k}$};
	
	% Draw example discretization regions
	    % Set parameters for the region to draw
	    \def\m{2} \def\n{1} \def\sigselmin{.15} \def\sigselmax{.25} 
	    % Call external (parametrized) tikz code
    	% Thought for use inside discretization_grid.tex only. Takes ``arguments'' from defined macros and
% plots the corresponding discretization region.	
	
	% Set PGF variables as mathematical expressions of other variables
	    \pgfmathsetmacro{\mpo}{\m+1} \pgfmathsetmacro{\npo}{\n+1}
	% Draw square on xy plane
	    \draw[dashed] (\xmin+\n*\xbynbins,0,\ymin+\m*\ybynbins) -- (\xmin+\npo*\xbynbins,0,\ymin+\m*\ybynbins) --
			(\xmin+\npo*\xbynbins,0,\ymin+\mpo*\ybynbins) -- (\xmin+\n*\xbynbins,0,\ymin+\mpo*\ybynbins) --
			(\xmin+\n*\xbynbins,0,\ymin+\m*\ybynbins);
	% Throw lines from the corners of the selected xy square
	    \draw[thin,dashed,gray!60] (\xmin+\n*\xbynbins,0,\ymin+\m*\ybynbins) -- (\xmin+\n*\xbynbins,\sigmaxn,\ymin+\m*\ybynbins);
	    \draw[thin,dashed,gray!60] (\xmin+\npo*\xbynbins,0,\ymin+\m*\ybynbins) -- (\xmin+\npo*\xbynbins,\sigmaxn,\ymin+\m*\ybynbins);
	    \draw[thin,dashed,gray!60] (\xmin+\npo*\xbynbins,0,\ymin+\mpo*\ybynbins) -- (\xmin+\npo*\xbynbins,\sigmaxn,\ymin+\mpo*\ybynbins);
	    \draw[thin,dashed,gray!60] (\xmin+\n*\xbynbins,0,\ymin+\mpo*\ybynbins) -- (\xmin+\n*\xbynbins,\sigmaxn,\ymin+\mpo*\ybynbins);
	% Draw selected segment on \sigma dimension
	    \draw[dashed] (.25*\xdist,\sigselmin*\lofsig,0) -- (.25*\xdist,\sigselmax*\lofsig,0);
	% Throw lines from the corners of the selected segment
	    \draw[thin,dashed,gray!60] (.25*\xdist,\sigselmin*\lofsig,0) -- (\xmax,\sigselmin*\lofsig,0);
	    \draw[thin,dashed,gray!60] (.25*\xdist,\sigselmax*\lofsig,0) -- (\xmax,\sigselmax*\lofsig,0);
	% Draw selected segment on \sigma-x plane
	    \draw[dashed] (\xmin+\n*\xbynbins,\sigselmin*\lofsig,0) -- (\xmin+\npo*\xbynbins,\sigselmin*\lofsig,0) -- (\xmin+\npo*\xbynbins,\sigselmax*\lofsig,0) -- (\xmin+\n*\xbynbins,\sigselmax*\lofsig,0) -- (\xmin+\n*\xbynbins,\sigselmin*\lofsig,0);
	% Throw lines upwards from the corners of the selected \sigma-x square
	    \draw[thin,dashed,gray!60] (\xmin+\n*\xbynbins,\sigselmin*\lofsig,0) -- (\xmin+\n*\xbynbins,\sigselmin*\lofsig,\ymax);
	    \draw[thin,dashed,gray!60] (\xmin+\npo*\xbynbins,\sigselmin*\lofsig,0) -- (\xmin+\npo*\xbynbins,\sigselmin*\lofsig,\ymax);
	    \draw[thin,dashed,gray!60] (\xmin+\npo*\xbynbins,\sigselmax*\lofsig,0) -- (\xmin+\npo*\xbynbins,\sigselmax*\lofsig,\ymax);
	    \draw[thin,dashed,gray!60] (\xmin+\n*\xbynbins,\sigselmax*\lofsig,0) -- (\xmin+\n*\xbynbins,\sigselmax*\lofsig,\ymax);
	% Highlight selected discretization region	
		% Foreground cube layer
		    \draw[thick] (\xmin+\n*\xbynbins,\sigselmin*\lofsig,\ymin+\m*\ybynbins) -- (\xmin+\n*\xbynbins,\sigselmin*\lofsig,\ymin+\mpo*\ybynbins) -- (\xmin+\n*\xbynbins,\sigselmax*\lofsig,\ymin+\mpo*\ybynbins) -- (\xmin+\n*\xbynbins,\sigselmax*\lofsig,\ymin+\m*\ybynbins) -- cycle;
		% Unseen cube lines
		    \draw[thick,dashed] (\xmin+\n*\xbynbins,\sigselmin*\lofsig,\ymin+\m*\ybynbins) -- (\xmin+\npo*\xbynbins,\sigselmin*\lofsig,\ymin+\m*\ybynbins) -- (\xmin+\npo*\xbynbins,\sigselmax*\lofsig,\ymin+\m*\ybynbins);
		    \draw[thick,dashed] (\xmin+\npo*\xbynbins,\sigselmin*\lofsig,\ymin+\m*\ybynbins) -- (\xmin+\npo*\xbynbins,\sigselmin*\lofsig,\ymin+\mpo*\ybynbins);
		% Lid cube layer
		    \draw[thick] (\xmin+\n*\xbynbins,\sigselmin*\lofsig,\ymin+\mpo*\ybynbins) -- (\xmin+\npo*\xbynbins,\sigselmin*\lofsig,\ymin+\mpo*\ybynbins) -- (\xmin+\npo*\xbynbins,\sigselmax*\lofsig,\ymin+\mpo*\ybynbins) -- (\xmin+\n*\xbynbins,\sigselmax*\lofsig,\ymin+\mpo*\ybynbins) -- cycle;
		% Remaining cube lines
		\draw[thick] (\xmin+\npo*\xbynbins,\sigselmax*\lofsig,\ymin+\mpo*\ybynbins) -- (\xmin+\npo*\xbynbins,\sigselmax*\lofsig,\ymin+\m*\ybynbins) --	(\xmin+\n*\xbynbins,\sigselmax*\lofsig,\ymin+\m*\ybynbins);
	    % Set parameters for the region to draw
	    \def\m{4} \def\n{4} \def\sigselmin{.75} \def\sigselmax{1} 
	    % Call external (parametrized) tikz code
	    % Thought for use inside discretization_grid.tex only. Takes ``arguments'' from defined macros and
% plots the corresponding discretization region.	
	
	% Set PGF variables as mathematical expressions of other variables
	    \pgfmathsetmacro{\mpo}{\m+1} \pgfmathsetmacro{\npo}{\n+1}
	% Draw square on xy plane
	    \draw[dashed] (\xmin+\n*\xbynbins,0,\ymin+\m*\ybynbins) -- (\xmin+\npo*\xbynbins,0,\ymin+\m*\ybynbins) --
			(\xmin+\npo*\xbynbins,0,\ymin+\mpo*\ybynbins) -- (\xmin+\n*\xbynbins,0,\ymin+\mpo*\ybynbins) --
			(\xmin+\n*\xbynbins,0,\ymin+\m*\ybynbins);
	% Throw lines from the corners of the selected xy square
	    \draw[thin,dashed,gray!60] (\xmin+\n*\xbynbins,0,\ymin+\m*\ybynbins) -- (\xmin+\n*\xbynbins,\sigmaxn,\ymin+\m*\ybynbins);
	    \draw[thin,dashed,gray!60] (\xmin+\npo*\xbynbins,0,\ymin+\m*\ybynbins) -- (\xmin+\npo*\xbynbins,\sigmaxn,\ymin+\m*\ybynbins);
	    \draw[thin,dashed,gray!60] (\xmin+\npo*\xbynbins,0,\ymin+\mpo*\ybynbins) -- (\xmin+\npo*\xbynbins,\sigmaxn,\ymin+\mpo*\ybynbins);
	    \draw[thin,dashed,gray!60] (\xmin+\n*\xbynbins,0,\ymin+\mpo*\ybynbins) -- (\xmin+\n*\xbynbins,\sigmaxn,\ymin+\mpo*\ybynbins);
	% Draw selected segment on \sigma dimension
	    \draw[dashed] (.25*\xdist,\sigselmin*\lofsig,0) -- (.25*\xdist,\sigselmax*\lofsig,0);
	% Throw lines from the corners of the selected segment
	    \draw[thin,dashed,gray!60] (.25*\xdist,\sigselmin*\lofsig,0) -- (\xmax,\sigselmin*\lofsig,0);
	    \draw[thin,dashed,gray!60] (.25*\xdist,\sigselmax*\lofsig,0) -- (\xmax,\sigselmax*\lofsig,0);
	% Draw selected segment on \sigma-x plane
	    \draw[dashed] (\xmin+\n*\xbynbins,\sigselmin*\lofsig,0) -- (\xmin+\npo*\xbynbins,\sigselmin*\lofsig,0) -- (\xmin+\npo*\xbynbins,\sigselmax*\lofsig,0) -- (\xmin+\n*\xbynbins,\sigselmax*\lofsig,0) -- (\xmin+\n*\xbynbins,\sigselmin*\lofsig,0);
	% Throw lines upwards from the corners of the selected \sigma-x square
	    \draw[thin,dashed,gray!60] (\xmin+\n*\xbynbins,\sigselmin*\lofsig,0) -- (\xmin+\n*\xbynbins,\sigselmin*\lofsig,\ymax);
	    \draw[thin,dashed,gray!60] (\xmin+\npo*\xbynbins,\sigselmin*\lofsig,0) -- (\xmin+\npo*\xbynbins,\sigselmin*\lofsig,\ymax);
	    \draw[thin,dashed,gray!60] (\xmin+\npo*\xbynbins,\sigselmax*\lofsig,0) -- (\xmin+\npo*\xbynbins,\sigselmax*\lofsig,\ymax);
	    \draw[thin,dashed,gray!60] (\xmin+\n*\xbynbins,\sigselmax*\lofsig,0) -- (\xmin+\n*\xbynbins,\sigselmax*\lofsig,\ymax);
	% Highlight selected discretization region	
		% Foreground cube layer
		    \draw[thick] (\xmin+\n*\xbynbins,\sigselmin*\lofsig,\ymin+\m*\ybynbins) -- (\xmin+\n*\xbynbins,\sigselmin*\lofsig,\ymin+\mpo*\ybynbins) -- (\xmin+\n*\xbynbins,\sigselmax*\lofsig,\ymin+\mpo*\ybynbins) -- (\xmin+\n*\xbynbins,\sigselmax*\lofsig,\ymin+\m*\ybynbins) -- cycle;
		% Unseen cube lines
		    \draw[thick,dashed] (\xmin+\n*\xbynbins,\sigselmin*\lofsig,\ymin+\m*\ybynbins) -- (\xmin+\npo*\xbynbins,\sigselmin*\lofsig,\ymin+\m*\ybynbins) -- (\xmin+\npo*\xbynbins,\sigselmax*\lofsig,\ymin+\m*\ybynbins);
		    \draw[thick,dashed] (\xmin+\npo*\xbynbins,\sigselmin*\lofsig,\ymin+\m*\ybynbins) -- (\xmin+\npo*\xbynbins,\sigselmin*\lofsig,\ymin+\mpo*\ybynbins);
		% Lid cube layer
		    \draw[thick] (\xmin+\n*\xbynbins,\sigselmin*\lofsig,\ymin+\mpo*\ybynbins) -- (\xmin+\npo*\xbynbins,\sigselmin*\lofsig,\ymin+\mpo*\ybynbins) -- (\xmin+\npo*\xbynbins,\sigselmax*\lofsig,\ymin+\mpo*\ybynbins) -- (\xmin+\n*\xbynbins,\sigselmax*\lofsig,\ymin+\mpo*\ybynbins) -- cycle;
		% Remaining cube lines
		\draw[thick] (\xmin+\npo*\xbynbins,\sigselmax*\lofsig,\ymin+\mpo*\ybynbins) -- (\xmin+\npo*\xbynbins,\sigselmax*\lofsig,\ymin+\m*\ybynbins) --	(\xmin+\n*\xbynbins,\sigselmax*\lofsig,\ymin+\m*\ybynbins);
	
	% Draw coordinate axis (x,\sigma,y)
	    \draw[thick,->] (0,0,0) -- (-9,0,0) node[anchor=north west]{$x$};
	    \draw[thick,->] (0,0,0) -- (0,5,0) node[anchor=south east]{$\dsigma$};
	    \draw[thick,->] (0,0,0) -- (0,0,5) node[anchor=south]{$y$};
\end{tikzpicture}